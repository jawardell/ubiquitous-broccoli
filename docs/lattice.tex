\documentclass[10pt,a4paper]{article}
\usepackage{verbatim}
\usepackage[latin1]{inputenc}
\usepackage{amsmath}
\usepackage{amsfonts}
\usepackage{amssymb}
\usepackage{graphicx}
\usepackage{pstricks, pst-all}
\begin{document}










\begin{comment}
\caption{Skewed-Top Corridor State Information}
  \begin{center}
    \begin{tabular}{ |c|c|c|c|c|c|c|c|c| } 
      \hline
        &    &   &    &    &   &   &    & \\ 
        &    &   &    &    &   &   & 0  &5 \\ 
        &    &   &    &    & 1 & 0 & 5  &0 \\ 
        &    &   &  0 &  1 & 0 & 5 & 0  &19 \\ 
        &  0 & 0 &  1 &  0 & 4 & 0 & 14 &0 \\ 
      0 &  0 & 1 &  0 &  3 & 0 & 9 & 0  &28 \\ 
      0 &  1 & 0 &  2 &  0 & 5 & 0 & 14 &0 \\ 
      1 &  0 & 1 &  0 &  2 & 0 & 5 & 0  &14 \\ 
      \hline
    \end{tabular}
  \end{center}
\end{comment}


%\begin{comment}
  \begin{center}
	  \caption{Reflected Skewed Top-Corridor}
  \psset{unit=0.5cm,xunit=.8cm}
  \begin{pspicture}(0,-10)(10,9)
  \psaxes[labels=none](0,0)(0,-8)(10,7)
  

    \psline[linestyle=dashed,linecolor=red](0,3)(11,8)
    \psline(0,0)(1,1)
    \psline(1,1)(2,2)
    \psline(2,2)(3,3)
    \psline(3,3)(4,4)

    \psline(1,1)(2,0)
    \psline(2,2)(3,1)
    \psline(3,3)(4,2)
    \psline(4,4)(5,3)

    \psline(2,0)(3,1)
    \psline(3,1)(4,2)
    \psline(4,2)(5,3)
    \psline(5,3)(6,4)
    \psline(6,4)(7,5)
    \psline(7,5)(8,6)


    \psline(3,1)(2,0)
    \psline(3,1)(4,0)
    \psline(4,2)(5,1)
    \psline(5,3)(6,2)
    \psline(6,4)(7,3)
    \psline(7,5)(8,4)
    \psline(8,6)(9,5)


    \psline(4,0)(5,1)
    \psline(5,1)(6,2)
    \psline(6,2)(7,3)
    \psline(7,3)(8,4)
    \psline(8,4)(9,5)
    \psline(9,5)(10,6)



    \psline(6,0)(7,1)
    \psline(7,1)(8,2)
    \psline(8,2)(9,3)
    \psline(9,3)(10,4)


    \psline(8,0)(9,1)
    \psline(9,1)(10,2)




    \psline(5,1)(6,0)
    \psline(6,2)(7,1)
    \psline(7,3)(8,2)
    \psline(8,4)(9,3)
    \psline(9,5)(10,4)

	  
    \psline(7,1)(8,0)
    \psline(8,2)(9,1)
    \psline(9,3)(10,2)
    \psline(9,1)(10,0)


%%%%%%%%%%%%%%%%%%%%%%%%%%%%%%%%%%%%%%%%%%%%%%%%%%%%
%%%%%%%%%%%%%%%%%%%%%%%%%%%%%%%%%%%%%%%%%%%%%%%%%%%%
%%%%%%%%%%%%%%%%%%%%%%%%%%%%%%%%%%%%%%%%%%%%%%%%%%%%

    \psline[linestyle=dashed,linecolor=red](0,-2)(10,-7)
    \psline(0,0)(1,-1)
    \psline(1,-1)(2,-2)
    \psline(2,-2)(3,-3)
    \psline(3,-3)(4,-4)

    \psline(1,-1)(2,0)
    \psline(2,-2)(3,-1)
    \psline(3,-3)(4,-2)
    \psline(4,-4)(5,-3)

    \psline(2,0)(3,-1)
    \psline(3,-1)(4,-2)
    \psline(4,-2)(5,-3)
    \psline(5,-3)(6,-4)
    \psline(6,-4)(7,-5)
    \psline(7,-5)(8,-6)


    \psline(3,-1)(2,-0)
    \psline(3,-1)(4,-0)
    \psline(4,-2)(5,-1)
    \psline(5,-3)(6,-2)
    \psline(6,-4)(7,-3)
    \psline(7,-5)(8,-4)
    \psline(8,-6)(9,-5)


    \psline(4,0)(5,-1)
    \psline(5,-1)(6,-2)
    \psline(6,-2)(7,-3)
    \psline(7,-3)(8,-4)
    \psline(8,-4)(9,-5)
    \psline(9,-5)(10,-6)



    \psline(6,0)(7,-1)
    \psline(7,-1)(8,-2)
    \psline(8,-2)(9,-3)
    \psline(9,-3)(10,-4)


    \psline(8,0)(9,-1)
    \psline(9,-1)(10,-2)




    \psline(5,-1)(6,0)
    \psline(6,-2)(7,-1)
    \psline(7,-3)(8,-2)
    \psline(8,-4)(9,-3)
    \psline(9,-5)(10,-4)

	  
    \psline(7,-1)(8,-0)
    \psline(8,-2)(9,-1)
    \psline(9,-3)(10,-2)
    \psline(9,-1)(10,0)
%%%%%%%%%%%%%%%%%%%%%%%%%%%%%%%%%%%%%%%%%%%%%%%%%%%%
%%%%%%%%%%%%%%%%%%%%%%%%%%%%%%%%%%%%%%%%%%%%%%%%%%%%
%%%%%%%%%%%%%%%%%%%%%%%%%%%%%%%%%%%%%%%%%%%%%%%%%%%%





    \psdots
	  (0,0)(0,1)(0,2)
	  (1,0)(1,2)(1,1)
	  (2,0)(2,2)(2,1)(2,3)
	  (3,0)(3,2)(3,1)(3,3)
	  (4,0)(4,2)(4,1)(4,3)(4,4)
	  (5,0)(5,2)(5,1)(5,3)(5,4)
	  (6,0)(6,2)(6,1)(6,3)(6,4)(6,5)
	  (7,0)(7,2)(7,1)(7,3)(7,4)(7,5)
	  (8,0)(8,2)(8,1)(8,3)(8,4)(8,5)(8,6)
	  (9,0)(9,2)(9,1)(9,3)(9,4)(9,5)(9,6)
	  (10,0)(10,2)(10,1)(10,3)(10,4)(10,5)(10,6)(10,7)


	\psdots
	  (1,0)(1,-2)(1,-1)
	  (2,0)(2,-2)(2,-1)(2,-3)
	  (3,0)(3,-2)(3,-1)(3,-3)
	  (4,0)(4,-2)(4,-1)(4,-3)(4,-4)
	  (5,0)(5,-2)(5,-1)(5,-3)(5,-4)
	  (6,0)(6,-2)(6,-1)(6,-3)(6,-4)(6,-5)
	  (7,0)(7,-2)(7,-1)(7,-3)(7,-4)(7,-5)
	  (8,0)(8,-2)(8,-1)(8,-3)(8,-4)(8,-5)(8,-6)
	  (9,0)(9,-2)(9,-1)(9,-3)(9,-4)(9,-5)(9,-6)
	  (10,0)(10,-2)(10,-1)(10,-3)(10,-4)(10,-5)(10,-6)(10,-7)

	  \begin{comment}
    \uput[70](0,0){\tiny{$1$}}
    \uput[70](0,1){\tiny{$0$}}
    \uput[70](0,2){\tiny{$0$}}

    \uput[u](1,0){\tiny{$0$}}
    \uput[u](1,1){\tiny{$1$}}
    \uput[u](1,2){\tiny{$0$}}

    \uput[u](2,0){\tiny{$1$}}
    \uput[u](2,1){\tiny{$0$}}
    \uput[u](2,2){\tiny{$1$}}

    \uput[u](3,0){\tiny{$0$}}
    \uput[u](3,1){\tiny{$2$}}
    \uput[u](3,2){\tiny{$0$}}

    \uput[u](4,0){\tiny{$2$}}
    \uput[u](4,1){\tiny{$0$}}
    \uput[u](4,2){\tiny{$2$}}


    \uput[u](5,0){\tiny{$0$}}
    \uput[u](5,1){\tiny{$4$}}
    \uput[u](5,2){\tiny{$0$}}

    \uput[d](-0.3,-0.3){\tiny $n = $}
    \uput[d](0,-0.3){\tiny $0$}
    \uput[d](1,-0.3){\tiny $1$}
    \uput[d](2,-0.3){\tiny $2$}
    \uput[d](3,-0.3){\tiny $3$}
    \uput[d](4,-0.3){\tiny $4$}
    \uput[d](5,-0.3){\tiny $5$}
	  \end{comment}



	  
%%%%%%%%%%%%%%%%%%%%%%%%%%%%%%%%%%%%%%%%%%%%%%%%%%%%

  %  \uput[l](-0.25,9){\tiny $k = 10$}
  %  \uput[l](-0.25,8){\tiny $k = 9$}
    \uput[l](-0.25,7){\tiny $k = 8$}
    \uput[l](-0.25,6){\tiny $k = 7$}
    \uput[l](-0.25,5){\tiny $k = 6$}
    \uput[l](-0.25,4){\tiny $k = 5$}
    \uput[l](-0.25,3){\tiny $k = 4$}
    \uput[l](-0.25,2){\tiny $k = 3$}
    \uput[l](-0.25,1){\tiny $k = 2$}
    \uput[l](-0.25,0){\tiny $k = 1$}
    \uput[l](-0.25,-1){\tiny $k = -1$}
    \uput[l](-0.25,-2){\tiny $k = -2$}
    \uput[l](-0.25,-3){\tiny $k = -3$}
    \uput[l](-0.25,-4){\tiny $k = -4$}
    \uput[l](-0.25,-5){\tiny $k = -5$}
    \uput[l](-0.25,-6){\tiny $k = -6$}
    \uput[l](-0.25,-7){\tiny $k = -7$}
    \uput[l](-0.25,-8){\tiny $k = -8$}
 %   \uput[l](-0.25,-9){\tiny $k = -9$}
%    \uput[l](-0.25,-10){\tiny $k = -10$}

    
%    \uput[d](8,-1){The Classical Skewed Top Corridor}


  \end{pspicture}
\end{center}
%\end{comment}

\end{document}
