\documentclass[10pt,a4paper]{article}
\usepackage{verbatim}
\usepackage[latin1]{inputenc}
\usepackage{xcolor}
\usepackage{amsmath}
\usepackage{amsfonts}
\usepackage{amssymb}
\usepackage{graphicx}
\usepackage{pstricks, pst-all}
\begin{document}

  \begin{center}
	  \caption{Reflected Skewed Top-Corridor}
  \psset{unit=0.5cm,xunit=.8cm}
  \begin{pspicture}(0,-20)(20,20)
  \psaxes[labels=none](0,0)(0,-8)(10,7)
  

    \psline[linestyle=dashed,linecolor=red](0,3)(12,9)
    \psline(0,1)(1,2)
    \psline(1,2)(2,3)
    \psline(2,1)(3,2)
    \psline(3,2)(4,3)
    \psline(4,3)(5,4)
    \psline(5,4)(6,5)
    \psline(4,1)(5,2)
    \psline(5,2)(6,3)
    \psline(6,3)(7,4)
    \psline(7,4)(8,5)
    \psline(8,5)(9,6)
    \psline(6,1)(7,2)
    \psline(7,2)(8,3)
    \psline(8,3)(9,4)
    \psline(9,4)(10,5)
    \psline(9,6)(10,7)
    \psline(8,1)(9,2)
    \psline(9,2)(10,3)
	


    \psline(1,2)(2,1)
    \psline(2,3)(3,2)
    \psline(3,2)(4,1)
    \psline(4,3)(5,2)
    \psline(5,2)(6,1)
    \psline(5,4)(6,3)
    \psline(6,3)(7,2)
    \psline(6,5)(7,4)
    \psline(7,2)(8,1)
    \psline(7,4)(8,3)
    \psline(8,3)(9,2)
    \psline(8,5)(9,4)
    \psline(9,2)(10,1)
    \psline(9,4)(10,3)
    \psline(9,6)(10,5)


%%%%%%%%%%%%%%%%%%%%%%%%%%%%%%%%%%%%%%%%%%%%%%%%%%%%
%%%%%%%%%%%%%%%%%%%%%%%%%%%%%%%%%%%%%%%%%%%%%%%%%%%%
%%%%%%%%%%%%%%%%%%%%%%%%%%%%%%%%%%%%%%%%%%%%%%%%%%%%

    
   \psline[linestyle=dashed,linecolor=red](0,-3)(10,-8)

    \psline(0,-1)(1,-2)
    \psline(1,-2)(2,-3)
    \psline(2,-1)(3,-2)
    \psline(3,-2)(4,-3)
    \psline(4,-3)(5,-4)
    \psline(5,-4)(6,-5)
    \psline(4,-1)(5,-2)
    \psline(5,-2)(6,-3)
    \psline(6,-3)(7,-4)
    \psline(7,-4)(8,-5)
    \psline(8,-5)(9,-6)
    \psline(6,-1)(7,-2)
    \psline(7,-2)(8,-3)
    \psline(8,-3)(9,-4)
    \psline(9,-4)(10,-5)
    \psline(9,-6)(10,-7)
    \psline(8,-1)(9,-2)
    \psline(9,-2)(10,-3)
	


    \psline(1,-2)(2,-1)
    \psline(2,-3)(3,-2)
    \psline(3,-2)(4,-1)
    \psline(4,-3)(5,-2)
    \psline(5,-2)(6,-1)
    \psline(5,-4)(6,-3)
    \psline(6,-3)(7,-2)
    \psline(6,-5)(7,-4)
    \psline(7,-2)(8,-1)
    \psline(7,-4)(8,-3)
    \psline(8,-3)(9,-2)
    \psline(8,-5)(9,-4)
    \psline(9,-2)(10,-1)
    \psline(9,-4)(10,-3)
    \psline(9,-6)(10,-5)
%%%%%%%%%%%%%%%%%%%%%%%%%%%%%%%%%%%%%%%%%%%%%%%%%%%%
%%%%%%%%%%%%%%%%%%%%%%%%%%%%%%%%%%%%%%%%%%%%%%%%%%%%
%%%%%%%%%%%%%%%%%%%%%%%%%%%%%%%%%%%%%%%%%%%%%%%%%%%%




    \psdots
	  (0,0)(0,1)(0,2)
	  (1,0)(1,2)(1,1)
	  (2,0)(2,2)(2,1)(2,3)
	  (3,0)(3,2)(3,1)(3,3)
	  (4,0)(4,2)(4,1)(4,3)(4,4)
	  (5,0)(5,2)(5,1)(5,3)(5,4)
	  (6,0)(6,2)(6,1)(6,3)(6,4)(6,5)
	  (7,0)(7,2)(7,1)(7,3)(7,4)(7,5)
	  (8,0)(8,2)(8,1)(8,3)(8,4)(8,5)(8,6)
	  (9,0)(9,2)(9,1)(9,3)(9,4)(9,5)(9,6)
	  (10,0)(10,2)(10,1)(10,3)(10,4)(10,5)(10,6)(10,7)

	\psdots
	  (1,0)(1,-2)(1,-1)
	  (2,0)(2,-2)(2,-1)(2,-3)
	  (3,0)(3,-2)(3,-1)(3,-3)
	  (4,0)(4,-2)(4,-1)(4,-3)(4,-4)
	  (5,0)(5,-2)(5,-1)(5,-3)(5,-4)
	  (6,0)(6,-2)(6,-1)(6,-3)(6,-4)(6,-5)
	  (7,0)(7,-2)(7,-1)(7,-3)(7,-4)(7,-5)
	  (8,0)(8,-2)(8,-1)(8,-3)(8,-4)(8,-5)(8,-6)
	  (9,0)(9,-2)(9,-1)(9,-3)(9,-4)(9,-5)(9,-6)
	  (10,0)(10,-2)(10,-1)(10,-3)(10,-4)(10,-5)(10,-6)(10,-7)


   
  




%%%%%%%%%%%%%%%%%%%%%%%%%%%%%%%
%%%%%%%%%%%%%%%%%%%%%%%%%%%%%%%
%%%%%%%%%%%%%%%%%%%%%%%%%%%%%%%
    \uput[70](0,0){\tiny{$1$}}
    \uput[70](0,1){\tiny{$0$}}
    \uput[70](0,2){\tiny{$0$}}

    \uput[u](1,0){\tiny{$0$}}
    \uput[u](1,1){\tiny{$1$}}
    \uput[u](1,2){\tiny{$0$}}

    \uput[u](2,0){\tiny{$1$}}
    \uput[u](2,1){\tiny{$0$}}
    \uput[u](2,2){\tiny{$1$}}

    \uput[u](3,0){\tiny{$0$}}
    \uput[u](3,1){\tiny{$2$}}
    \uput[u](3,2){\tiny{$0$}}

    \uput[u](4,0){\tiny{$2$}}
    \uput[u](4,1){\tiny{$0$}}
    \uput[u](4,2){\tiny{$2$}}


    \uput[u](5,0){\tiny{$0$}}
    \uput[u](5,1){\tiny{$4$}}
    \uput[u](5,2){\tiny{$0$}}

    \uput[d](-0.3,-0.3){\tiny $n = $}
    \uput[d](0,-0.3){\tiny $0$}
    \uput[d](1,-0.3){\tiny $1$}
    \uput[d](2,-0.3){\tiny $2$}
    \uput[d](3,-0.3){\tiny $3$}
    \uput[d](4,-0.3){\tiny $4$}
    \uput[d](5,-0.3){\tiny $5$}



	  
%    \uput[d](8,-1){The Classical Skewed Top Corridor}


  \end{pspicture}
\end{center}
%\end{comment}

\end{document}
