\documentclass{article}

\usepackage[utf8]{inputenc}
\usepackage[english]{babel}
\usepackage{amssymb}
\usepackage{amsfonts}
\usepackage{amsmath}
\usepackage{amsthm}
\usepackage{setspace}
\usepackage{amsthm}
\usepackage[margin=0.5in]{geometry}
\usepackage{pstricks, pst-all}
\usepackage{graphicx}

\linespread{1.25}

\newcommand\tab[1][1cm]{\hspace*{#1}}
\newcommand*{\set}[1]{\mathbb{#1}}
\topmargin = -0.75in

\newtheorem{thm}{Theorem}
\theoremstyle{definition}
\newtheorem{defn}[thm]{Definition}

\title{An Exploration of Skewed Top Corridors\vspace{-2ex}}
\author{Joanne Wardell}
\date{2017-2018\vspace{-2ex}}

\begin{document}
  
  \maketitle

  \tab Lattice paths have been studied extensively over the course of several centuries. For the purpose of this study, 
  a lattice consists of points $\set{Z}^2$ with certain restrictions and with only two allowable moves, up-right 
  and down right. Movements in various directions on the lattice are called paths.\par
  
  \tab We are studying a lattice path model which consists of a point at which all paths begin, upper and lower bounds 
  which mark areas beyond which non-zero paths cease to exist, and an ending point which marks the end of non-zero path  
  values. The area of the lattice contained by the upper and lower bound is referred to as a corridor.  The number of  
  paths within the corridor depend on the initial value which is placed at the starting position of the corridor, the 
  nature of the upper and lower bounds, and the position of the starting value in the corridor. The upper and lower 
  boundary lines mark the areas where non-zero path values cease to exist. Paths can exist on the two boundary lines,  
  but they cannot exist beyond the lines. The lower boundary is essentially a horizontal line with zero slope. The upper 
  boundary line is a line with variable slope. These conditions seem to present a problem when attempting to 
  systematically generate values contained in the corridors. Due to the nature of the upper bound, which is a line 
  with slope not parallel to that of the lower bound, the paths bounce off of the upper diagonal line, rippling into 
  and distorting the data below it. One might propose that calculating the error caused by each interruption would be 
  somewhat intuitive, but the impacts of each diagonal-boundary disturbance grow larger as time in the corridor 
  progresses. Although the data changes because of the upper bound's slope, intriguing patterns and characteristics 
  have been observed in the configurations of this environment. The corridor model which has varying boundary lines 
  is referred to as a skewed top corridor.\par
  
  \tab We interpret the corridor, for now, on the two dimensional Cartesian coordinate plane which is the span of unit  
  vectors $\hat{\imath} = \vec(1, 0)$ and $\hat{\jmath} = \vec(0, 1)$. 
  We refer to the vertical axis as the y-axis and the horizontal axis as the x-axis.
  The elapse of time is represented along the x-axis. 
  The amount of paths is represented along the y-axis which is perpendicular to the x-axis. At a glance, the structure 
  of the corridor consists of an initial value placed at a starting position in the corridor. A gap of predetermined size 
  exists in the same column as the starting position. The size of the gap is equal to the number of lattice points extending    
  vertically from the starting position and includes the starting path value. The gap, for now, extends vertically in the   
  positive direction of the y-axis, for which our convention is upward. Values in this gap are zeros except for the starting 
  value which is also contained in the gap. Our exploration consists of manipulating the initial conditions of the corridors, 
  observing and quantifying the results, and assembling various formulas to express relationships that we notice. We want to 
  learn as much as we can about these corridor structures and what the relationships within them might encode.\par
  %might delete last sentence
  
  \subsection*{Symbols and Definitions}
  \tab We start our journey of exploring skewed top corridors by establishing our definitions. We also need a symbolic method 
  for referring to specific aspects of the corridors. We will use these specific symbols and definitions not only to discuss 
  corridors but to also make mathematical conclusions about the structures.\\*\\*
  
  \noindent First off, we will be using the set of numbers which are integers larger than one.
  \begin{defn} Let the set of natural numbers be represented by $\set{N}$.\\ \end{defn}\par
  
  \noindent We will also discuss integers which are positive and greater than zero. 
  \begin{defn} Let $\set{N}_0$ represent the set of natural numbers including zero. \end{defn}
  
  %margin chops this part off.. :'(
  \noindent At times, we will need to utilize the set of integer values which are positive, negative, and include zero. 
  \begin{defn} Let $\set{Z}$ represent the set of integers. \\ \end{defn}
  
  \noindent We can now begin defining some of the parameters for establishing the skewed-top corridor. Let us begin by defining 
  the corridor gap. This will be an important feature later in our journey. 
  \begin{defn} Let $g \in \set{N}$ equal the gap size in the corridor. \end{defn}
  \noindent This value $g$ is the number of points in the column of the initial starting path value in the corridor. It 
  simply counts the number of points in the starting column including the starting value. The gap is one of the parameters 
  which is required to establish a skewed top corridor. 
  %might not put this here
  We will allow the gap to vary over time and will later discuss this in more detail.\\
  
  \noindent Another important parameter for the skewed top corridor is the value that is placed at the 
  beginning of the structure. This value exists at a position in the gap. We will define this position momentarily.
  \begin{defn} Let $v \in \set{N}$ represent the initial number of paths at the beginning of the corridor.\\ \end{defn}
  
  \noindent The position at which the initial value $v$ is placed is an important factor. 
  \begin{defn} Let $a \in \set{N}$ be the location on the horizontal axis where the paths begin. The starting point 
  will always be $(a, 0)$.\\ \end{defn}

  \noindent The corridor containing all possible path combinations is a subset on an infinite plane of integers. Consider a function 
  whose input is two natural numbers, including zero, and whose output is a natural number or zero. 
  \begin{defn} Let $V \mathpunct{:} \set{N}_0 \times \set{N}_0 \rightarrow \set{N}_0$ and let us refer to this as a corridor state 
  function. The output of $V$ is $\in \set{N}_0$ and denotes the number of paths at some point in the corridor.\\ \end{defn}
  
  \noindent As we vary these parameters, different corridors are produced. We've developed the following notation for characterizing 
  corridors.
  \begin{defn} Let $C_g$ be the corridor with gap $=g$. \\ \end{defn}
  
  \noindent The diagonal values in said corridors are one of the features of interest. Diagonals are characterized by their start and end points.
  We will learn more about the characteristics of corridor diagonals and will develop more definitions later in the paper. 
  \begin{defn} Let $D_y$, $y \in \set{Z}$, be the diagonal beginning at position $(0, y)$ and ending at an inclusive point on corridor's upper boundary. \\ \end{defn}

  
  
  \subsection*{Propagating Values}
  \tab When initializing the skewed top corridor and its values, it is important to follow a specific algorithm. The following is an ordered description of 
  setting up the corridor boundaries, placing initial values, and generating paths in the corridor structure.\\*
  \tab Initialize all numbers on the plane to zero. Choose some value for $a$ and some value for $v$. For initial analysis, we choose to set static values for $
  v$ and $a$. Assume, for now, $v = 1$ and $a = 1$.
  Insert the initial value $v$ into position $(0, a)$, that is, place an initial value of $1$ into position $(0,1)$ on the plane.\\*
  \tab Choose some value for $g$, which represents the size of a gap in the first corridor column. More specifically, $g$ is an indication of where the upper 
  boundary of the corridor begins. If the corridor were placed in an array, the gap column would consist of $g$ entries: the initial value, and $g-1$ zeros above $v$.
  On the plane, the gap begins at inclusive point $(0,2)$ and extends upward to non-inclusive coordinate $(0, g+1)$. A sentinel marker is placed at $(0, g+1)$ 
  indicating the first establishment of the corridor's upper boundary. From now on, the upper boundary of the corridor may be referred to as the sentinel line.\\*
  \tab Assign a value to $m = \frac{1}{r}$. For now, let $r = 2$, so that $m = \frac{1}{2}$. From the initial sentinel point at $(0, g+1)$, continue establishing 
  such points according to the following procedure. Continually move one unit rightward, keeping track of the current column number, or, horizontal coordinate. 
  Record the current position on the vertical axis as well. If the current column is divisible by $r$, place a sentinel marker at the $(horizontal, vertical)$ 
  position on the grid, move up one unit on the horizontal axis, and move rightward one unit on the vertical axis. If the current column is not divisible by $r$, 
  do not place a sentinel marker, move up zero units on the horizontal axis, and move rightward one unit on the vertical axis. Do this until the current vertical 
  position is $n$.\\*
  Generally, the procedure for initializing the sentinel line is as follows:
  \begin{itemize}
    \item[] place first sentinel point at $(0, g+1)$ 
    \item[] while current vertical is $<$ $n$,
    \item[] \tab increase horizontal position by one
    \item[] \tab note current position $(horizontal, vertical)$
    \item[] \tab\tab if current vertical position $mod$ $r = 0$
    \item[] \tab\tab\tab place sentinel marker at current position
    \item[] \tab\tab\tab increase vertical position by one
  \end{itemize}

  Another way to express this is $(x,y) \in \set{N}_0^2$ with $0 < y < mx+g_1$. Since $m=\frac{1}{2}$, move up one row in the plane when an even column
  is encountered. This is continued until one reaches the $n^{th}$ column. A sentinel line is
  established with a slope of $\frac{1}{2}$. For now, assume that every value on or above the sentinel line 
  is equal to zero and assume that every value below the lower boundary is equal to zero. The following recursive formula can be used to
  calculate each value in the array, where $(0, k) = $ initial state:
	\begin{align}
		V(x,y) = V(x-1, y-1) + V(x-1,y+1)\text{ , }
		x \in \set{N} \text{ , } 0 < y < mx+g+1\text{ , }
		V(0,y) =  \begin{cases}
			v \text{, if }y=a\\
			0 \text{, otherwise}
		\end{cases}
	\end{align}
  
 
  
  \subsection*{Observations}
    \noindent The corridor will contain values beginning at $(0,1)$, with rows starting from $1$ extending to $g+ \frac{n+1}{2}$ inclusive
    and columns starting from $0$ extending to $n$ inclusive.\\*\\*
    
    \noindent The sentinel points implicitly form a line with a slope of $m$, which we have set to $\frac{1}{2}$. The following statements 
    build upon the characteristics of this line, the sentinel line.\\*\\*

    \noindent From point $(0, 1)$, a line of $1$s with slope one is generated until the sentinel line is encountered at $(2g-1, 2g)$.
    Building on the idea of where the line of ones will encounter the sentinel line, it can be proven that the line of ones will always contain $2g$ paths and will always 
    intersect the sentinel line at $(2g-1, 2g+1)$.\\*
    The line of ones will exist due to the nature of how the numbers in the corridor are generated, see (1). The value $V(k,x)$ accepts an upper contribution of $0$ and a
    lower contribution from the initial value $1$. Because of this initial state, a line of $1$s will be generated because numbers outside of the corridor are $0$ and 
    contributions from the previous $1$ will calculate more $1$s. An equation for the sentinel line is $y = \frac{1}{2}x+g+1$ and an equation for the line of ones 
    is $y=x+1$. To see where these two lines intersect, simply set the two equations equal to each other and solve for the $x$ and $y$ coordinates of the intersection.\\
    \[\frac{1}{2}x+g+1=x+1\]
    \[x = 2g\]
    \[y=2g+1\]
    \begin{align}(2g, 2g+1) \end{align} \\
    
    \noindent The $D_{-1}$ diagonal is the counting numbers.
 By setup and $(1)$, $D_{1}$ is a sequence consisting of $2g$ elements.
 The following is an inductive proof based on the fact that the line of $1$s exists and formula (1).

\begin{gather*}
  \text{Prove   } V(k+1, k) = k\text{ , where } 2 \leq k \leq 2g\text{.} \\
  \text{Base Case: }k = 2\\
  V(k+1,k) = V(k,k-1) + V(k,k+1)\\
  V(3,2) = V(2,1) + V(2,3)\\
  2 = 2\\
  \text{Assume } V(k+1, k) = k \text{ for all } \leq k \leq 2g \text{. }
  \text{Use } k+1 \text{.}\\
  V(k+2,k+1)\\
  V(k+1,k) + V(k+1, k+2)
\end{gather*}
\begin{align}
  \tab\tab\tab\tab\tab\tab k + 1^* \tab\tab\tab \text{ *while } k \leq 2g
 \end{align}
    
  \subsection*{Conjecture}
  
  \tab Four-element arithmetic sequences exist in the corridors along with additional sequences of greater length. If $D_{-3}$ is an 
  arithmetic sequence with degree $N$, where $N \in \set{N}_0$, then
  $D_{-1}$ is a sequence with degree $N+1$ and $D_{-1}$ is a sequence with degree $N-1$. Because the values within the corridors are defined by formula $(1)$, 
  $D_{1}$ is a degree $0$ sequence by setup, and $D_{-1}$ is a degree $1$ sequence by $(2)$,
  sequences contribute to each other linearly via addition. This results in sequences and subsequences in the corridor that are of increasing degree 
  as one approaches $-\infty$ on the $y$ axis.\\
  \tab $2g$-element sequences begin at coordinate $(0,2y-1)$, end at coordinate $(2g-1, 2g-|y|)$, and have degree $N = -\frac{1}{2}(y - 1)$. 
  For $(2y-1) < 1$, the first instance four-element subsequences begins at coordinate $(2g, 2g-y)$, ends at coordinate $(2g + 3, 2g-y + 3)$, and is of degree
  $N = -\frac{1}{2}(y + 1)$.\\
  \tab Subsequence $\sigma_{\hat{y}}^{\hat{x}}$, where ${\hat{y} \in \set{Z}}$ and ${\hat{x} \in \set{N}_0}$, denotes the $\hat{x}^{th}$ 
  subsequence of diagonal $D_{\hat{y}}$. Four-element subsequences of $D_{\hat{y}}$ are classified as all $\sigma_{\hat{y}}^{\hat{x}}$ where 
  $\hat{x} > 0$. For all $4$-element subsequences after $\hat{x} = 1$, the beginning coordinate is $(2g + 3\hat{x}, 2g+3\hat{x})$ and 
  the ending coordinate is $(2g + 3\hat{x}+3, 2g+3\hat{x}+3)$. 
  $\sigma_{\hat{y}}^{\hat{x}}$ with $\hat{x} > 0$ have a degree of $N = -\frac{1}{2}(\hat{y} - 1) - \hat{x}$.
  


 
  
  
  
  

		
  
  



\end{document}


