\documentclass{article}

\usepackage[utf8]{inputenc}
\usepackage[english]{babel}
\usepackage{amssymb}
\usepackage{amsfonts}
\usepackage{amsmath}
\usepackage{amsthm}
\usepackage{setspace}
\usepackage{amsthm}
\usepackage[margin=0.5in]{geometry}

\linespread{1.25}

\newcommand\tab[1][1cm]{\hspace*{#1}}
\newcommand*{\set}[1]{\mathbb{#1}}
\topmargin = -1.75in

\newtheorem{thm}{Theorem}
\theoremstyle{definition}
\newtheorem{defn}[thm]{Definition}

\title{An Exploration of Skewed Top Corridors\vspace{-2ex}}
\author{Joanne Wardell}
\date{2017-2018\vspace{-2ex}}

\begin{document}
  
  \maketitle

  \tab Lattice paths have been studied extensively over the course of several centuries. For the purpose of this study, 
  a lattice consists of points $\set{Z}^2$ with certain restrictions and with only two allowable moves, up-right 
  and down right. Movements in various directions on the lattice are called paths.\par
  
  \tab We are studying a lattice path model which consists of a point at which all paths begin, upper and lower bounds 
  which mark areas beyond which non-zero paths cease to exist, and an ending point which marks the end of non-zero path  
  values. The area of the lattice contained by the upper and lower bound is referred to as a corridor.  The number of  
  paths within the corridor depend on the initial value which is placed at the starting position of the corridor, the 
  nature of the upper and lower bounds, and the position of the starting value in the corridor. The upper and lower 
  boundary lines mark the areas where non-zero path values cease to exist. Paths can exist on the two boundary lines,  
  but they cannot exist beyond the lines. The lower boundary is essentially a horizontal line with zero slope. The upper 
  boundary line is a line with variable slope. These conditions seem to present a problem when attempting to 
  systematically generate values contained in the corridors. Due to the nature of the upper bound, which is a line 
  with slope not parallel to that of the lower bound, the paths bounce off of the upper diagonal line, rippling into 
  and distorting the data below it. One might propose that calculating the error caused by each interruption would be 
  somewhat intuitive, but the impacts of each diagonal-boundary disturbance grow larger as time in the corridor 
  progresses. Although the data changes because of the upper bound's slope, intriguing patterns and characteristics 
  have been observed in the configurations of this environment. The corridor model which has varying boundary lines 
  is referred to as a skewed top corridor.\par
  
  \tab We interpret the corridor, for now, on the two dimensional Cartesian coordinate plane which is the span of unit  
  vectors $\hat{\imath} = \vec(1, 0)$ and $\hat{\jmath} = \vec(0, 1)$. The elapse of time is represented along the x-axis.  
  The amount of paths is represented along the y-axis which is perpendicular to the x-axis. At a glance, the structure 
  of the corridor consists of an initial value placed at a starting position in the corridor. A gap of predetermined size 
  exists in the same column as the starting position. The size of the gap is equal to the number of lattice points extending    
  vertically from the starting position and includes the starting path value. The gap, for now, extends vertically in the   
  positive direction of the y-axis, for which our convention is upward. Values in this gap are zeros except for the starting 
  value which is also contained in the gap. Our exploration consists of manipulating the initial conditions of the corridors, 
  observing and quantifying the results, and assembling various formulas to express relationships that we notice. We want to 
  learn as much as we can about these corridor structures and what the relationships within them might encode.\par
  %might delete last sentence
  
  \subsection*{Symbols and Definitions}
  \tab We start our journey of exploring skewed top corridors by establishing our definitions. We also need a symbolic method 
  for referring to specific aspects of the corridors. We will use these specific symbols and definitions not only to discuss 
  corridors but to also make mathematical conclusions about the structures.\\*\\*
  
  \noindent First off, we will be using the set of numbers which are integers larger than one.
  \begin{defn} Let the set of natural numbers be represented by $\set{N}$.\\ \end{defn}\par
  
  \noindent We will also discuss integers which are positive and greater than zero. 
  \begin{defn} Let $\set{N}_0$ represent the set of natural numbers including zero. \end{defn}
  
  %margin chops this part off.. :'(
  \noindent At times, we will need to utilize the set of integer values which are positive, negative, and include zero. 
  \begin{defn} Let $\set{Z}$ represent the set of integers. \\ \end{defn}
  
  \noindent We can now begin defining some of the parameters for establishing the skewed-top corridor. Let us begin by defining 
  the corridor gap. This will be an important feature later in our journey. 
  \begin{defn} Let $g \in \set{N}$ equal the gap size in the corridor. \end{defn}
  \noindent This value $g$ is the number of points in the column of the initial starting path value in the corridor. It 
  simply counts the number of points in the starting column including the starting value. The gap is one of the parameters 
  which is required to establish a skewed top corridor.

  
\end{document}


